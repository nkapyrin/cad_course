%%% Здесь выбираются необходимые графы
%\documentclass[russian,utf8,pointsection,nocolumnsxix,nocolumnxxxi,nocolumnxxxii]{eskdtext}
\documentclass[russian,pointsection,12pt]{eskdtext}
\usepackage{fontspec}

\usepackage{hyperref}
%\usepackage[hyphens]{hyperref}

\usepackage{listings}

% Устранение расстояний между элементами списков
\usepackage{enumitem}
\setlist{nosep}
\usepackage{setspace}

\newcommand{\No}{\textnumero}
\newcommand{\DecimalNumber}{}
%\usepackage{showframe}

%%% Чтобы работал eskdx и некоторые другие пакеты LaTeX
\usepackage{xecyr}

\usepackage{eskdchngsheet}
\usepackage{eskdtotal}

%%% Для работы шрифтов
\usepackage{xunicode,xltxtra}

%%% Для работы с русскими текстами (расстановки переносов, последовательность комманд строго обязательна)
\usepackage{polyglossia}
\setdefaultlanguage{russian}

% Ставим ГОСТ как основной шрифт
\usepackage{xcolor}
\usepackage{fontspec}
% [Path={fonts/}, BoldFont={GOST_type_A-Bold}, ItalicFont={GOST_type_A-Italic}]

\setmainfont{GOST_type_A}
\setsansfont{GOST_type_A}
\setmonofont{GOST_type_A}
\newfontfamily{\cyrillicfont}{GOST_type_A}[Path={fonts/}, UprightFont={GOST_type_A}, BoldFont={GOST_type_A_Bold}, ItalicFont={GOST_type_A_Italic}]
\newfontfamily{\cyrillicfontt}{GOST_type_A}[Path={fonts/}, UprightFont={GOST_type_A}, BoldFont={GOST_type_A_Bold}, ItalicFont={GOST_type_A_Italic}]
\newfontfamily{\cyrillicfonttt}{GOST_type_A}[Path={fonts/}, UprightFont={GOST_type_A}, BoldFont={GOST_type_A_Bold}, ItalicFont={GOST_type_A_Italic}]

% Ставим LiberationSerif как основной шрифт
%\newfontfamily{\cyrillicfont}{LiberationSerif}
%\newfontfamily{\cyrillicfontt}{LiberationSans}
%\newfontfamily{\cyrillicfonttt}{LiberationMono}

%\defaultfontfeatures{Mapping=tex-text}

%%% Для работы со сложными формулами
\usepackage{amsmath}
\usepackage{amssymb}

%%% Чтобы использовать символ градуса (°) - \degree
\usepackage{gensymb}

%%% Для переноса составных слов
%\XeTeXinterchartokenstate=1
\XeTeXcharclass `\- 24
\XeTeXinterchartoks 24 0 ={\hskip\z@skip}
\XeTeXinterchartoks 0 24 ={\nobreak}

%%%% Ставим подпись к рисункам. Вместо «рис. 1» будет «Рисунок 1»
%\addto{\captionsrussian}{\renewcommand{\figurename}{Рисунок}}
%%%% Убираем точки после цифр в заголовках
%\def\russian@capsformat{%
%  \def\postchapter{\@aftersepkern}%
%  \def\postsection{\@aftersepkern}%
%  \def\postsubsection{\@aftersepkern}%
%  \def\postsubsubsection{\@aftersepkern}%
%  \def\postparagraph{\@aftersepkern}%
%  \def\postsubparagraph{\@aftersepkern}%
%}

% Автоматически переносить на след. строку слова которые не убираются
% в строке
\sloppy

%%% Для вставки рисунков
\usepackage{graphicx}

%%% Для вставки ссылок на файлы, интернет ссылок, полезно в библиографии
%\usepackage{url}
%\usepackage[hyphens]{url}
% Шрифт ссылок внутри документа должен оставаться текстовым, с засечками
\urlstyle{rm}
\DeclareUrlCommand\path{\urlstyle{rm}}
\DeclareUrlCommand\pathtt{\urlstyle{tt}}
%\DeclareUrlCommand\path{\def\UrlBreaks\urlstyle{rm}}
%\DeclareUrlCommand\pathtt{\def\UrlBreaks\urlstyle{tt}}


%%% Подподразделы(\subsubsection) не выводим в содержании
\setcounter{tocdepth}{2}

%%% Каждый раздел (\section) с новой страницы
\let\stdsection\section
\renewcommand\section{\newpage\stdsection}

%%% В введении нумерация подразделов идёт с буквой «В» (например В.1)
\makeatletter
\renewcommand\thesubsection{\ifnum\c@section=0{В.\arabic{subsection}}\else{\arabic{section}.\arabic{subsection}}\fi}
\makeatother

\renewcommand{\ESKDsectionAlign}{\ESKDsectAlignCenter}

%%% Новые стили для ЕСПД

\setlength{\headsep}{10mm}%
%\setlength{\voffset}{20mm}%

%\usepackage[numbertop]{eskdplain}
\ESKDnewStyle{espdtitle}{15mm}
\ESKDputOnStyle{espdtitle}{stamp}{%
	\newsavebox{\ESKDcolumnsxixboxX}
	\savebox{\ESKDcolumnsxixboxX}{
	\setlength{\unitlength}{1mm}%
	\begin{picture}(0,0)(0,0)
	\linethickness{\ESKDlineThick}
	\put(0, 0){\line(1,0){145}}
	\put(0, 7){\line(1,0){145}}
	\put(0, 12){\line(1,0){145}}
	\put(0, 0){\line(0,1){12}}
	\put(145,0){\line(0,1){12}}
	\linethickness{\ESKDlineThin}
	\put(25, 0){\line(0,1){12}}
	\put(60, 0){\line(0,1){12}}
	\put(85, 0){\line(0,1){12}}
	\put(110, 0){\line(0,1){12}}
	\put(0, 8.3){\makebox[25mm]{\ESKDfontIII\ESKDcolumnXIXname}}
	\put(0, 2.3){\makebox[25mm]{\ESKDfontIII\ESKDtheColumnXIX}}
	\put(25, 8.3){\makebox[35mm]{\ESKDfontIII\ESKDcolumnXXname}}
	\put(60, 8.3){\makebox[25mm]{\ESKDfontIII\ESKDcolumnXXIname}}
	\put(60, 2.3){\makebox[25mm]{\ESKDfontIII\ESKDtheColumnXXI}}
	\put(85, 8.3){\makebox[25mm]{\ESKDfontIII\ESKDcolumnXXIIname}}
	\put(85, 2.3){\makebox[25mm]{\ESKDfontIII\ESKDtheColumnXXII}}
	\put(110, 8.3){\makebox[35mm]{\ESKDfontIII\ESKDcolumnXXIIIname}}
	\end{picture}}
	
	\put(\ESKDltu{\ESKDframeX},\ESKDltu{\ESKDframeY}){%
		\begin{turn}{90}\usebox{\ESKDcolumnsxixboxX}\end{turn}
	}
}
\ESKDdefaultTitleStyle{espdtitle}

% Формат первой и остальных страниц
\ESKDnewStyle{espdpage}{15mm}
\ESKDputOnStyle{espdpage}{frame}{%
	\put(7, 285){\makebox[210mm]{\thepage}}
	\put(7, 278){\makebox[210mm]{\DecimalNumber}}
	%\setlength{\headsep}{10mm}%
	%\setlength{\headheight}{5mm}%
	%\setlength{\topmargin}{0mm}%
}
\ESKDdefaultFirstStyle{espdpage}
\ESKDdefaultStyle{espdpage}

% Шрифт большинства элементов титульного листа
\renewcommand{\ESKDfontVsize}{\fontsize{14pt}{16pt}}
% Основной шрифт будет прямым (по умолчанию, наклонный)
\renewcommand{\ESKDfontShape}{\normalfont}

\clubpenalty =10000


\makeatletter
\newcommand*{\lstlinelink}[2]{%
  \refused{#1}%
  \edef\lstlinelink@tmp{\getrefbykeydefault{#1}{anchor}{}}%
  \ifx\lstlinelink@tmp\@empty
    #2%
  \else
    \edef\lstlinelink@target{%
      \expandafter\lstlinelink@parse\lstlinelink@tmp\@nil
      #2%
    }%
    \hyperlink{\lstlinelink@target}{#2}%
  \fi
}
\def\lstlinelink@parse#1.#2\@nil{lstnumber.#2.}%
\makeatother

\renewcommand\lstlistingname{Листинг}
\lstset{
    language=C++,
    basicstyle=\linespread{0.4}\small,
    numbers=left,
    numberstyle=\tiny\color{gray},
    keepspaces=true,
    columns=flexible,
    breaklines=true,
    captionpos=t,
    frame=tb
}








